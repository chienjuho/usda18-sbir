\section{Research Team and Plan}
\label{sec:plan}

Here, we describe the research team that will
perform the investigations we propose, and then follow that with
the our collaboration plan and timeline for Phase~I along with
our current thoughts concerning the Phase~II effort.

\subsection{Team}

Organized in 1991, BECS is privately held and employs approximately 100
people. It has outgrown its current
manufacturing facility of 24,000 sq.~ft.\ so is in the process of
renovating and moving into a new 42,000 sq.~ft.\ facility.
Its intellectual property is protected with 12 issued patents and
2 patents pending.

The design engineering team at BECS is comprised of 10 individuals, of
which Todd Steinbrueck is the software team lead (who will serve as
PI on the SBIR grant) and Roger Chamberlain is the VP Engineering (who
will serve as senior personnel on the SBIR grant).\footnote{Roger Chamberlain
also is Professor of Computer Science and Engineering at Washington
University in St.~Louis. While bringing over 25 years of research
experience to bear on the problem at hand, Dr.~Chamberlain will be
serving in his capacity as VP Engineering at BECS and will maintain
strict compliance with the conflict-of-commitment rules at the the university.}
Additional members of the design engineering team will contribute to
the effort as needed. 
%Mr.~Steinbrueck and Dr.~Chamberlain will each
%commit \roger{XX} of time to the project, and BECS will dedicate
%an additional
%one FTE of software engineering effort (as articulated in the budget).

This team has extensive experience in control equipment design and
implementation, including custom implementation of both the hardware and
the software. Presently, BECS designs and manufactures equipment for
animal husbandry (the focus of this proposal),
e.g., grain bin content monitoring,
feed auger controls, environmental monitoring and control of animal houses;
maintains proper water chemistry in aquatics systems;
monitors and controls
refrigeration in commercial settings, e.g., valve controllers that adapt
automatically to different refrigerants;
and delivers CATV signals, e.g., RF amplifiers, line extenders.

BECS provides equipment to the agriculture industry under a private-label
arrangement with Grain Systems, Inc. (GSI), a subsidiary of
AGCO Corporation,
which is a publicly traded company (NYSE:AGCO) with net sales
of \$7.4~billion in 2016.
GSI provides BECS equipment (and equipment manufactured by itself and others)
to the market
through both its Cumberland (poultry) and AP (swine) divisions.

As the vision of IoT becomes a reality, equipment providers
generally need to evolve so that they partner with end users
(farmers in our case) to become a service company as well as a
manufacturing company. BECS and GSI both aspire to this goal.
Quoting from AGCO's
website,\footnote{\url{www.agcocorp.com/brands/gsi.html}}
``Grain, swine, and poultry producers need more than equipment,
they need full-scale solutions that boost overall performance
and productivity.''

As described in Section~\ref{sec:background}, BECS has supported
remote connectivity to its equipment for years.  What is new is that
rather than simply providing access to data (and expecting end users
to make use of it on their own), BECS is increasingly finding ways to
help end users by being actively involved in data analysis.
While initial connectivity solutions
merely supported pull semantics (i.e., users had to
manually connect to the equipment to
see what was going on), current systems actively contact users when
conditions warrant (e.g., alarm conditions, feed stocks depleted),
pushing information to farmers on demand.
Through the combination of functionality within the control equipment 
and data services provided via the cloud, the vision is one in
which animal husbandry is simpler to execute,
more efficient,
less resource intensive (both in terms of human resources and
feed and water resources), and results in healthier animals.

%\roger{fix for Ag.}
%
%The existing business model is primarily based upon equipment sales.
%Annual revenues for FY14 through FY16 are \$9.9 million, \$10.7 million,
%and \$11.8 million, respectively
%(with \$13 million predicted for FY17), across all the markets BECS
%serves.\footnote{These revenue numbers are proprietary information.}
%To sustain the level of service we aspire to deliver, there must be
%a revenue stream to support the required effort.  Our vision here is
%a subscription model, where end users pay a monthly fee for the
%top tier service model.
%This revenue stream is already a reality in the agricultural (Ag) market.
%For a monthly fee, data from our Ag equipment is collected automatically
%(by BECS), retained in the cloud, and made available to the appropriate
%owner/operator (i.e., farmer).  Absent that monthly subscription,
%owners/operators of the Ag equipment are free to collect their data
%themselves, directly from the controller.
%We anticipate the ability of BECS to provide privacy-preserving,
%aggregated data access to aquatic control equipment to be well worth
%the cost of a monthly subscription.

BECS has received government funding on two separate occasions
(see Section~\ref{sec:prior} for details), both of which were SBIR/STTR
grants through the National Institutes of Health.  We currently
have a pending application with the National Science Foundation
that focuses on similar issues as this proposal but is targeted at
the aquatics industry.

We will collaborate on this SBIR project with Dr.~Alvitta Ottley,
Asst.~Professor of Computer Science and Engineering at Washington University
in St.~Louis, and Dr.~C.-J. Ho, also
Asst.~Professor of Computer Science and Engineering at Washington University
in St.~Louis.
In this proposed agenda, Dr. Alvitta Ottley will lead the design
of the visualization front-end of the system that will allow clients
to interact with the data.
Dr.~Ottley has built and evaluated a range of visualization tools
and designs~\cite{brown2014finding, hakone2017proact,ottley2015personality,peck2013using}. 
Relevant to the proposed work, her prior research has focused on designing visualizations to decision support~\cite{hakone2017proact,ottley2012visually,ottley2016improving} for non-experts. 
Her work has also made significant advancement toward evaluating the effectiveness of visualization designs~\cite{peck2013using,ziemkiewicz2013visualization}, and understanding how users interactions with visualization tools~\cite{brown2014finding,ottley2015personality}.  
%Dr.~Ottley will commit \roger{XX} of effort towards the project
%in Phase~I.

Dr.~Ho will investigate the design of incentives to motivate farmers to
contribute their data. Dr. Ho's research has focused on the learning and incentive
problems in human-in-the-loop systems. In particular, he has worked on designing
reputation systems and monetary contracts to encourage online users
to contribute high-quality data~\cite{HSV14,HZVS12}
and measuring how users respond to incentives in 
real-world applications~\cite{hssv15}.


We will consult on the project with Dr.~Liyue Fan, Asst.~Professor
of Information Security and Digital Forensics, University at Albany (SUNY).
Dr.~Fan's research interests are in data privacy, spatiotemporal data
analysis, and database applications.  Her research is one of two
currently viable approaches to differentially private time series data,
and is the one we plan to focus on in this project.
Dr.~Fan will commit 2~weeks of effort towards the project.

Together the proposal team has the necessary knowledge and experience to undertake the proposed work. 
More specific information regarding coordination and collaboration plans is
provided below.

\subsection{Research Plan}

The primary participants in the project all reside in St. Louis, Missouri,
so face-to-face meetings have a relatively low overhead. We will coordinate
the project through periodic (e.g., biweekly) meetings that alternate
locations (one meeting at BECS Technology's facility, and two weeks later
another meeting on campus at Washington Univ. in St. Louis).

Dr. Fan is physically remote (she is in Albany, NY), so her primary
method of interaction will be via electronic communication.  We will
use video-conferencing fairly extensively when face-to-face discussions
are warranted.  In addition, the Dept.~of Computer Science and Engineering
at Washington University in St.~Louis
will invite Dr.~Fan to present a colloquium on campus at or very near
the beginning of the project.  This will enable all of the parties to
have a true face-to-face interaction without any travel expense charged
to the grant.

A formal agreement will need to be put in place between the company and
Washington~U.; however, the two organizations have collaborated in the
past (on a previous STTR grant through the National Institutes of Health).
The terms of that previous agreement will form the
basis of the required new agreement, and we anticipate no difficulties
in finalizing the agreement itself.

The proposed research is suited for an eight-month research agenda.
The schedule is shown in Figure~\ref{fig:gantt}.
The various research components can initally be independently investigated,
primarily in parallel by the two organizations, coming together
in the assessments to be performed at the end of Phase~I.

\begin{figure}[ht]
\centering
\setlength\tabcolsep{2pt}
\begin{tabular}{l l | c c c c c c c c}
\  & & \multicolumn{8}{c}{Month}\\
\textbf{Task} & & 1 &  2 & 3 & 4 & 5 & 6 & 7 & 8 \\ \hline
\textbf{Data aggregation} & \S\ref{sec:da} & & & & & & & & \\
-- {\small Implementation} &  & $\blacksquare$ & $\blacksquare$ & $\blacksquare$ & $\blacksquare$ & & & &  \\
-- {\small Exploration of parameter space} &  & & & $\blacksquare$ & $\blacksquare$ & $\blacksquare$ & $\blacksquare$ & &  \\
-- {\small Utility assessment} & & & & & & $\blacksquare$ & $\blacksquare$ & $\blacksquare$ & $\blacksquare$  \\ \hline
\textbf{Formulating incentives} & \S\ref{sec:fi} & & & & & & &  \\
-- {\small Experiment design} & & $\blacksquare$ & $\blacksquare$ & $\blacksquare$ & & & &  \\
-- {\small Experiment execution} & & & & & $\blacksquare$ & $\blacksquare$ & $\blacksquare$ & &  \\
-- {\small Results evaluation} & & & & & & & & $\blacksquare$ & $\blacksquare$  \\ \hline
\textbf{Visualization} & \S\ref{sec:vis} & & & & & & &  \\
-- {\small Experiment design} & & $\blacksquare$ & $\blacksquare$ & $\blacksquare$ & & & &  \\
-- {\small Experiment execution} & & & & & $\blacksquare$ & $\blacksquare$ & $\blacksquare$ & &  \\
-- {\small Results evaluation} & & & & & & & & $\blacksquare$ & $\blacksquare$  \\ \hline
\textbf{Empirical evaluation} & \S\ref{sec:exp} & & & & & & & & \\
-- {\small Customer recruitment} & & & & & & $\blacksquare$ & $\blacksquare$ & & \\ 
-- {\small Customer evaluation} & & & & & & & & $\blacksquare$ & $\blacksquare$ \\
\end{tabular}
    \caption{Project schedule, eight month duration.}
    \label{fig:gantt}
\end{figure}

The results of the assessments will inform
the activities to be pursued during Phase~II.
If the result of Phase~I is data that are appropriately anonymized,
one can consider the focus of Phase~II to be the determination of
what it is we can effectively learn from the anonymized data.

The learning from data will come in two forms:
\begin{itemize}
\item {\bf Machine learning} -- we will investigate utilizing techniques
described in the literature~\cite{acgmmtz16,fs10,ss15} for performing machine
learning on differentially private data sets.
We will apply techniques such as these to the problem of barn
control, with the goal of diminishing feed consumption, providing
tighter control, and improving feed conversion rates.
\item {\bf Human learning} -- we will explore ways in which we can effectively
communicate to laypersons an intuitive notion of ``privacy budget,'' and
how to visualize risk, which has been studied in the
management and
financial space~\cite{Eppler09,Sarlin16}, but we are unaware of any
work concerning privacy risk.
\end{itemize}

In addition to exploring what can be learned from data in a privacy
preserving manner, there are a number of followup investigations that
will be initiated during Phase~I but will require additional effort
during Phase~II.  First, our initial efforts will be focused on time-series
data; however, event data are also present in our data sets and we will
need to investigate techniques for including them in properly anonymized
form.

Second, both our incentive and visualization investigations during
Phase~I will be restricted to participants we recruit via Amazon's
Mechanical Turk.  This gives us flexible and inexpensive access to
willing participants, with a broad set of backgrounds.  However, we are
particularly interested in the impact of these investigations on our target
audience, farmers.  The Phase~II effort will include targeted studies
to address this issue.

Third, the incentive mechanisms during Phase~I will be limited to monetary
incentives.  As we discover what can be learned from the aggregated data,
it is entirely possible that simply access to this information is a
sufficient incentive for farmers to be willing to participate.  This will
also be a part of our Phase~II investigation.
